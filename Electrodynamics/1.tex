\RequirePackage{luatex85}
\documentclass[leqno]{ltjsarticle}
\usepackage{luatexja-fontspec}
\usepackage[top=10truemm,bottom=10truemm,left=20truemm,right=20truemm]{geometry}
\usepackage{luatexja} 
\usepackage{multicol,amsmath,amssymb,mathtools,ascmac,amsthm,amscd,physics,comment,dcolumn,titlesec,mathrsfs,mypkg,tikz-cd,tensor}
\usetikzlibrary{arrows.meta}
\titleformat*{\section}{\Large\bfseries}
\setlength{\parindent}{0pt}
\pagestyle{empty}
%\everymath{\displaystyle}
\begin{document}
	\begin{equation}
		\tensor{\eta}{_\mu_\nu} \coloneq
		\begin{pmatrix}
			1 & 0 & 0 & 0\\
			0 & -1 & 0 & 0\\
			0 & 0 & -1 & 0\\
			0 & 0 & 0 & -1
		\end{pmatrix}
	\end{equation}

\begin{equation}
\tensor{F}{_\mu_\nu}\coloneq
	\begin{pmatrix}
		0 & E_1 & E_2 & E_3\\
		-E_1 & 0 & -B_3 & B_2\\
		-E_2 & B_3 & 0 & -B_1\\
		-E_3 & -B_2 & B_1 & 0
	\end{pmatrix}\ \  
\tensor{F}{^\mu^\nu}=
	\begin{pmatrix}
		0 & -E_1 & -E_2 & -E_3\\
		E_1 & 0 & -B_3 & B_2\\
		E_2 & B_3 & 0 & -B_1\\
		E_3 & -B_2 & B_1 & 0
	\end{pmatrix}
\end{equation}
	This tensor is called Faraday tensor and staisfies the following equation.
	\begin{equation}
		\partial^\gamma\tensor{F}{^\beta^\nu} + \partial^\beta\tensor{F}{^\nu^\gamma} + \partial^\nu\tensor{F}{^\gamma^\beta} = 0 
	\end{equation}
	\begin{shadebox}
	This means 
	\begin{equation}
		\left\{\,
			\begin{aligned}
				&\nabla\times E= -\pdv{B}{t} \\
				&\nabla\cdot B = 0
			\end{aligned}
		\right.
	\end{equation}
\end{shadebox}
	
	\begin{equation}
		\mathcal{J}^{\nu} \coloneq \frac{1}{\mu_0}\partial_{\mu}\tensor{F}{^\mu^\nu}
	\end{equation}

	\begin{shadebox}
		If you denote $\mathcal J = {}^t(\rho,J) = {}^t(\rho,j_x,j_y,j_z)$
		\begin{equation}
			\left\{\,
				\begin{aligned}
					&\nabla\cdot E = \mu_0\rho\\
					&\nabla\times B = \mu_0J + \pdv{E}{t}
				\end{aligned}
			\right.
		\end{equation}
	\end{shadebox}
	\begin{equation}
		\tensor{\mathcal T}{^\mu^\nu} \coloneq \frac{1}{\mu_0}\qty(\tensor{F}{^\mu^\alpha}\tensor{\eta}{_\alpha_\beta}\tensor{F}{^\beta^\nu} + \frac{1}{4}\tensor{\eta}{^\mu^\nu}\tensor{F}{^\beta^\gamma}\tensor{F}{_\beta_\gamma}) 
	\end{equation}
		This tensor is called energy-momentum tensor.
	\begin{itembox}[l]{Thm}
		\begin{equation}
			\partial_\mu \tensor{\mathcal T}{^\mu^\nu} = \mathcal J_{\mu}\tensor{F}{^\mu^\nu}
		\end{equation}
	\end{itembox}
	\begin{align*}
		\partial_\mu \tensor{\mathcal T}{^\mu^\nu}&= \frac{1}{\mu_0}\qty(\partial_{\mu}\tensor{F}{^\mu^\alpha}\tensor{\eta}{_\alpha_\beta}\tensor{F}{^\beta^\nu} + \frac{1}{4}\partial_{\mu}\tensor{\eta}{^\mu^\nu}\tensor{F}{^\beta^\gamma}\tensor{F}{_\beta_\gamma})
	\end{align*}
\end{document}

