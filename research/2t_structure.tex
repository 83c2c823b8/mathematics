\section{t-structure}
\begin{defn}\cite{BBD}
		$\D$を三角圏.直和因子と同型を取る操作で閉じている加法部分圏$\D^{\le 0},\D^{\ge 0}\subset\D$が次の条件を満たすとき,$(\D^{\le 0},\D^{\ge 0})$を$\D$のt-構造と呼ぶ.
		$\D^{\le n}\coloneq\D^{\le 0}[-n],\hspace{3mm}\D^{\ge n}\coloneq\D^{\ge 0}[-n] $
		\begin{itemize}
			\item[(i)]
				$\D^{\le 0}\subset\D^{\le 1}\hspace{3mm}\D^{\ge 1}\subset\D^{\ge 0} $
			\item[(ii)]
				$\D^{\ge 1}\subset (\D^{\le 0})^\perp$\\
				つまり,$\forall E\in \D^{\le 0},\ \forall F\in\D^{\ge 1},\ \Hom_{\D}(E,F)=0$
			\item[(iii)]
				任意の$E\in\D$に対して
				\[\tau_{\le 0}E\rightarrow E \rightarrow \tau_{\ge 1}E\rightarrow \tau_{\le 0}E[1]\]
				となるような$\tau_{\le 0}E\in\D^{\le 0},\ \tau_{\ge 1}E\in\D^{\ge 1}$が存在する.
		\end{itemize}
\end{defn}

\begin{lemm}
		\begin{itemize}
				\item[(i)]$\D^{\ge 1}=(\D^{\le 0})^{\perp}$
				\item[(ii)]$E\in\D$に対して,$\tau_{\le 0}E\in\D^{\le 0}, \tau_{\ge 1}E\in\D^{\ge 1}$は同型を除いて一意に定まる.
		\end{itemize}
\end{lemm}
	\begin{proof}
		$E\in\mathcal (\D^{\le 0})^\perp$ を任意にとる.定義より以下の完全三角形がとれる.
				\[\tau_{\le 0}E\rightarrow E \rightarrow \tau_{\ge 1}E\rightarrow \tau_{\le 0}E[1]\]
				定義より,$\tau_{\le 0}E\rightarrow E$は0射であるので,$\tau_{\ge 1}E\simeq E\oplus\tau_{\le 0}E[1]\in \D^{\ge 1}$.直和因子と同型を取る操作で閉じているので$E\in\D^{\ge 1}$.\\
				一意性については,別の$F\in\D^{\le 0},\ G\in\D^{\ge 1}$と完全三角形
				\[F\rightarrow E \rightarrow G\rightarrow F[1]\]
				があったとすると,$G[-1]\in\D^{\ge 2}$なので$\Hom_{\D}(F,G[-1])=\Hom_{\D}(F,G)=0$なので,
\end{proof}
\begin{defn}\cite{BBD}
	$(D^{\le 0},\D^{\ge 0})$がt-構造を与えるとき,$\D$の部分圏$\H\coloneq\D^{\le 0}\cap\D^{\ge 0}$はt-構造の核(the heart of t-structure)と呼ばれる.
\end{defn}

\begin{lemm}\cite{BBD}
	$\H$は核,余核をもつ.
\end{lemm}
\begin{proof}
	$E,F\in\H$,$f\in\Hom_{\H}(E,F)$を任意にとる.このとき,三角圏の公理より以下の完全三角形
	\[C\xrightarrow{e}A\xrightarrow{f}B\xrightarrow{g}C[1]\]
	が存在する.t-構造の定義より
	\[X\xrightarrow{x}C[1]\xrightarrow{y}Y\rightarrow X[1]\]
	が完全三角形となるような$X\in \D^{\le -1},Y\in \D^{\ge 0}$が存在する.このとき,
	\[y\circ g\colon B\rightarrow Y\]
	が$f$の余核を与えることを示す.
	\begin{comment}
			\[
				\begin{tikzcd}[column sep=huge,row sep =huge]
					B[-1] \ar[r,"g\texttt{[-1]}"]\ar[d,equal]& C\ar[r,]\ar[d,"y\texttt{[-1]}",swap]& A\ar[d,"\ell",dotted] \ar[r,"f"]& B\ar[d,equal]\\
					B[-1] \ar[r,"y\texttt{[-1]}\circ g\texttt{[-1]}"]\ar[d,"g\texttt{[-1]}",swap]& Y[-1]\ar[r,]\ar[d,equal]& M\ar[d,dotted] \ar[r,"m"]& B\ar[d,"g"]\\
					C \ar[r,]\ar[d,swap]& Y[-1]\ar[r,]\ar[d,swap]& X \ar[r,"x"]\ar[d,equal]& C[1] \ar[d,]\\
			A \ar[r,"\ell",dotted]& M\ar[r,dotted]&  X\ar[r,dotted]& A[1]\\
		\end{tikzcd}
			\]
\end{comment}	
			定義より,$A\in\H\subset\D^{\le 0}$,$X\in\D^{\le -1}\subset\D^{\le 0}$であり,八面体公理より存在する以下の完全三角形から
	\[A\rightarrow M\rightarrow X\rightarrow A[1]\]
	$M\in\D^{\le 0}$がわかる.$B\in\H\subset\D^{\le 0}$,$A[1]\in \D^{\le -1}\subset \D^{\le 0}$より$C[1]\in\D^{\le 0}$がしたがう.$X[1]\in\D^{\le -2}\subset\D^{\le 0}$と合わせて,$Y\in\D^{\le 0}$がしたがう.取り方により$Y\in\D^{\ge 0}$だったので$Y\in\H$がわかる.\\
	$Q\in\H$と$q\circ f$を満たす$q\in\Hom_{\H}(B,Q)$を任意にとると,$q=q'\circ g$を満たす$q'\in\Hom_{\D}(C[1],Q)$が存在する.また,$X\in\D^{\le -1}$,$Q\in\D^{\ge 0}$なので$g\circ q'=0$.したがって,$q'=y\circ q''$を満たす$q''\in\Hom_{\D}(Y,Q)$が存在する.完全系列
	\[A\rightarrow B\xrightarrow{g} C[1]\rightarrow A[1]\]
	に対して,コホモロジー的函手$\Hom_{\D}(-,Q)$を作用させると$A[1]\in\D^{\le -1}$なので$\Hom_{\D}(A[1],Q)=0$から完全列
	\[0\rightarrow \Hom_{\D}(C[1],Q) \rightarrow \Hom_{\D}(B,Q)\]
	が存在して,$q=q'\circ g$となる$q'$の一意性がわかる.どうように$X[1]\in\D^{\le -1}$なので,
	\[0\rightarrow \Hom_{\D}(Y[1],Q) \rightarrow \Hom_{\D}(C[1],Q)\]
	$q'=q''\circ y$となる$q''$の一意性がわかる.したがって,$y\circ g\colon B\to Y$が$f\colon A\to B$の余核を与えていることが示された.反対圏を考えることで核の存在も証明される.
\end{proof}

\begin{thm}\cite{BBD}
	$\H$はアーベル圏となる.
\end{thm}
\begin{proof}
	核,余核の存在は示されたので任意の射$f\in\Hom_{\H}(A,B)$にたいして,
\[\Im(f)\simeq \Coim(f)\]
の存在を言えばよい.
\end{proof}

\begin{lemm}
	$\D$を三角圏,$\A\subset\D$を充満部分加法圏としたとき,$\A$有界なt-構造$\F\subset\D$の核であることと以下の条件が同値.\vspace{-3mm}
	\begin{enumerate}[label=\roman*.]
		\item
			任意の$k_1,k_2\in\ZZ\ (k_1>k_2)$と$A,B\in\A$に対して,$\Hom_{\D}(A[k_1],B[k_2])=0$
		\item
			任意の対象$E\in\D$に対して,有限の整数の列
			\[k_1>k_2 \cdots > k_n\]
			が存在して$A_j\in \A[k_j]$となるような分解が存在する.
	\end{enumerate}
\end{lemm}
\begin{proof}
	\[
		\begin{tikzcd}[column sep=1.3em]
			0\ar[r,equal]&E_0\ar[rr]& & E_{1}\ar[r]\ar[ld] &\cdots\ar[r] &E_{n-2}\ar[rr]&&E_{n-1}\ar[ld]\ar[rr]&&E_n\ar[ld]\ar[r,equal]&E\\
									 &&A_{1}\ar[lu,dotted,"{[1]}"] &&&&A_{n-1}\ar[lu,dotted,"{[1]}"]&&A_n\ar[lu,dotted,"{[1]}"]
		\end{tikzcd}
	\]
				
	\end{proof}
