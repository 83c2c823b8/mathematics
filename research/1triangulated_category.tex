\section{三角圏の公理}
\begin{defn}
	以下の条件をみたす圏$\C$を前加法圏(preadditive category)という.
	\vspace{-3mm}
	\begin{itemize}
		\item[(i)]
			任意の$X,Y\in\C$に対して,射の集合$\Hom_{\C}(X,Y)$がアーベル群になる.
		\item[(ii)]
			任意の$X,Y,Z\in\C$に対して,合成写像
		\[
			\begin{array}{ccccc}
				\circ\colon\Hom_{\C}(Y,Z) \times \Hom_{\C}(X,Y) & \longrightarrow & \Hom_{\C}(X,Z) \\
				\rotatebox{90}{\in}& &\rotatebox{90}{\in}\\
															(g, f) & \longmapsto & g \circ f
					\end{array}
\]
が双線型である.つまり,任意の$g,g'\in\Hom_{\C}(Y,Z),\ f,f'\in\Hom_{\C}(X,Y)$に対して,
\begin{gather}
	(g+g')\circ f = g\circ f + g'\circ f\\
	g\circ(f+f') = g\circ f + g\circ f'
\end{gather}
	\end{itemize}
が成り立つ.
\end{defn}

\begin{defn}
	以下の条件をみたす前加法圏$\C$を加法圏(additive category)という.
	\vspace{-3mm}
	\begin{itemize}
	\item[(i)]零対象(始対象かつ終対象)である$0\in\C$をを持つ.
	\item[(ii)]任意の対象$X,Y\in\C$に対し直和$X\oplus Y$が存在する.
	\end{itemize}
	\vspace{-3mm}
\end{defn}

\begin{defn}
	 	加法圏$\C,\D$に対し,函手$F\colon\C\to\D$が以下の条件を満たすとき加法函手(additve functor)という.
		\begin{itemize}
			\item[(1)]
				任意の$X,Y\in\C$に対し,
				\[
			\begin{array}{ccccc}
				F\colon\Hom_{\C}(X,Y) & \longrightarrow & \Hom_{\D}(F(X),F(Y)) \\
				\rotatebox{90}{\in}& &\rotatebox{90}{\in}\\
															f & \longmapsto & F(f)
					\end{array}
				\]
				がアーベル群の準同型写像になっている.つまり$f,g \in\Hom_{\C}(X,Y)$に対し,
				\[F(f + g) = F(f) + F(g)\]
				となる.
			\item[(2)]
				$0\in\C$に対し,$F(0)=0$
			\item[(3)]
				直和を保つ.
				\[F(A\oplus B)\simeq F(A)\oplus F(B)\]
		\end{itemize}
\end{defn}

\begin{defn}
	$\C$を加法圏とする.このとき,$f\in\Hom_{\C}(X,Y)$の核(kernel)とは$\Ker f\in\C$と$\ker f\in\Hom_{\C}(\Ker f,X)$の組$(\Ker f,\ker f)$であって,以下の普遍性をみたすものである.
	\[\begin{tikzcd}
		K\ar[rd,"k"]\ar[d,"\exists !h",dotted,swap]\\
		\Ker f\ar[r,"\ker f",swap]&X\ar[r,"f",swap]&Y
\end{tikzcd}\]
\begin{itemize}
	\item[(i)]
		$f\circ\ker f=0$
	\item[(ii)]
		任意の$K\in\C$と$k\in\Hom_{\C}(K,X)$で$f\circ k=0$を満たすものに対して,一意に$h\in\Hom_{\C}(K,\Ker f)$が存在して,$\ker f\circ h=k$となる.\\
$f\in\Hom_{\C}(X,Y)$の$\C$での反対圏での核を余核(cokernel)とよび,$(\Cok f,\cok f)$と記す.
\end{itemize}
\end{defn}

\begin{defn}
	加法圏$\A$がアーベル圏(Abelian category)であるとは以下を満たす場合である.
	\vspace{-3mm}
	\begin{itemize}
		\item[(i)]
			$\A$の任意の射$f$に対し,核$\Ker f$と余核$\Cok f$が存在する.
		\item[(ii)]
			$\A$の任意の射$f$に対し,自然な同型\ $\Coim(f)\simeq\Im f$が存在する.
	\end{itemize}
\end{defn}

\begin{defn}\cite{KS06}
	$\D$を加法圏,$[1]$を自己同値函手とする.
	完全三角形と呼ばれる$\D$における三角形の集合を備えた,以下の性質をみたす組$(\D,[1])$を三角圏と呼ぶ$$$$
	\vspace{-3mm}
	\begin{itemize}
		\item[(i)]
			任意の$X\in\D$に対して,
			\[X\xrightarrow{\id_X}X\rightarrow 0 \rightarrow X[1]\]
			は完全三角形.
		\item[(ii)]
			\[
		\begin{tikzcd}
			X_1\ar[r]\ar[d,"f"]& X_2\ar[r]\ar[d,"g"]& X_3\ar[r]\ar[d,"h"] & X_1[1]\ar[d,"f\texttt{[1]}"]\\
			Y_1\ar[r]& Y_2\ar[r]& Y_3\ar[r] & Y_1[1]\\
		\end{tikzcd}
			\]
			上の可換図式において,$f,g,h$が同型で$X_1\rightarrow X_2\rightarrow X_3 \rightarrow X_1[1]$が完全三角形なら$Y_1\rightarrow Y_2\rightarrow Y_3 \rightarrow Y_1[1]$も完全三角形
		\item[(iii)]
			任意の$X\xrightarrow{f}Y$は
			\[X\xrightarrow{f} Y\rightarrow Z \rightarrow X[1]\]
		と完全三角形に拡張できる.
	\item[(iv)]
		\[
			X_1\xrightarrow{u} X_2\xrightarrow{v} X_3\xrightarrow{w}  X_1[1]
	\]
	が完全三角形であることと
	\[
		X_2\xrightarrow{v} X_3\xrightarrow{w} X_1[1]\xrightarrow{-u[1]}  X_2[1]
	\]
	が完全三角形であることが同値.
	\item[(v)]
		\[
		\begin{tikzcd}
			X_1\ar[r]\ar[d,"f"]& X_2\ar[r]\ar[d,"g"]& X_3\ar[r]\ar[d,"h",dotted] & X_1[1]\ar[d,"f\texttt{[1]}"]\\
			Y_1\ar[r]& Y_2\ar[r]& Y_3\ar[r] & Y_1[1]\\
		\end{tikzcd}
	\]
	2つの完全三角形と図式を可換にする$f,g$が存在したとき,すべての四角形を可換にする$h$が存在する.

	\item[(vi)]
		八面体公理\\
			\[
				\begin{tikzcd}[row sep=5pt]
			E \ar[r,"f"]& F\ar[r,"h"]& X \ar[r]& E[1]\\
			E \ar[r,"g\circ f"]& G\ar[r,"l"]& Y \ar[r]& E[1]\\
			F \ar[r,"g"]& G\ar[r,"k"]& Z \ar[r]& F[1]\\
		\end{tikzcd}
			\]
			上記の3つの完全三角形に対して,以下の図式のすべての四角形を可換にし,4行目を完全三角形にするような$u\in\Hom_{\D}(X,Y),v\in\Hom_{\D}(Y,Z),w\in\Hom_{\D}(Z,X[1])$が存在する.
			\[
		\begin{tikzcd}
			E \ar[r,"f"]\ar[d,equal]& F\ar[r,"h"]\ar[d,"g",swap]& X\ar[d,"u",dotted] \ar[r]& E[1]\ar[d,equal]\\
			E \ar[r,"g\circ f"]\ar[d,"f",swap]& G\ar[r,"l"]\ar[d,equal]& Y\ar[d,"v",dotted] \ar[r]& E[1]\ar[d,"f\texttt{[1]}"]\\
			F \ar[r,"g"]\ar[d,"h",swap]& G\ar[r,"k"]\ar[d,"l",swap]& Z \ar[r]\ar[d,equal]& F[1]\ar[d,"h\texttt{[1]}"]\\
			X \ar[r,"u",dotted]& Y\ar[r,"v",dotted]& Z \ar[r,"w",dotted]& X[1]\\
		\end{tikzcd}
			\]
	\end{itemize}
\end{defn}

\begin{exmp}
	アーベル圏$\A$に対して,その導来圏$\D(\A)$は自然な三角圏の構造をもつ.また,有界導来圏$\D^b(\A)$も三角圏の構造をもつ.
\end{exmp}

\begin{prop}\cite{KS06}
	$\A$をアーベル圏とする.\\
	$\A$はその導来圏$\D(\A)$および有界導来圏$\D^b(\A)$の部分圏として自然に埋め込まれる.
\end{prop}


\begin{defn}
	アーベル圏$\A$と三角圏$\D$に対して,Grothendieck群$K(\A)$と$K(\D)$をそれぞれ対象たちで自由生成された群を以下の群で割ったものであると定める.
	\[\langle[Y]-[X]-[Z]\mid 短完全列\ 0\rightarrow X\rightarrow Y\rightarrow Z\rightarrow 0\ が\A で存在する.\rangle\]
	\[\langle [Y]-[X]-[Z]\mid 完全三角形\  X\rightarrow Y\rightarrow Z\rightarrow X[1]\ が\D で存在する.\rangle\]
\end{defn}

\begin{lemm}
	三角圏$\D$において,任意の射$f\colon E\rightarrow F$
	\[E\xrightarrow{f} F\xrightarrow{g} G\rightarrow E[1]\]
\end{lemm}

\begin{defn}
	三角圏$\D$からアーベル圏$\A$への加法函手$\H\colon \D\to\A$がコホモロジー的であるとは以下の条件を満たすときである.\\
	$\D$における任意の完全三角形
	\[X\xrightarrow{f}Y\xrightarrow{g}Z\xrightarrow{h}X[1]\]
	に対して,$\A$において以下の列が完全列になる
	\[\H(X)\xrightarrow{\H(f)}\H(Y)\xrightarrow{\H(g)}\H(Z)\]
\end{defn}

\begin{defn}
	$\D$を三角圏,$\C$をその充満部分加法圏としたとき以下の用語を定義する.
	\begin{itemize}
		\item[(i)]
			$\C$の$\D$における右(左)直交部分圏(right (left) orthogonal subcategory) $\C^\perp,{}^\perp\C$を$\D$の充満部分圏として
			\[\C^\perp\coloneq \{Y\in\D\mid\forall X\in\C,\ \Hom_{\D}(X,Y)=0\}\]
			\[{}^\perp\C\coloneq \{X\in\D\mid\forall Y\in\C,\ \Hom_{\D}(X,Y)=0\}\]
		\item[(ii)]
			$\C$が$\D$の狭義充満部分圏(strictly full subcategory)であるとは以下を満たすときである.\\
			$X\in\C$かつ$Y\in\D$に対して$X\simeq Y$ならば$Y\in\C$
		\item[(iii)]
			$\C$の$\D$におけるthick閉包とは直和因子を取る操作で閉じている最小の狭義充満部分三角圏のことである.それを$\thick \C$と記す.
		\item[(iv)]
			$\C$がthick部分圏(thick subcategory)であるとは,$\C = \thick\C$が成り立つときをいう.
	\end{itemize}

\end{defn}
$X^\bullet\in\D^b(\A)$に対して,$\tau_{\le i},\ \tau_{> i},\sigma_{\le i}, \sigma_{> i}$を
\begin{gather*}
	\tau_{\le i}X^\bullet\coloneq (\cdots \rightarrow X^{i-2}\rightarrow X^{i-1}\rightarrow \Ker d_X^i\rightarrow 0\rightarrow \cdots)\\
	\tau_{> i}X^\bullet\coloneq (\cdots \rightarrow 0\rightarrow \Im d_X^i\rightarrow X^{i+1}\rightarrow X^{i+2}\rightarrow \cdots)\\
	\sigma_{\le i}X^\bullet\coloneq (\cdots \rightarrow X^{i-2}\rightarrow X^{i-1}\rightarrow X^i\rightarrow 0\rightarrow \cdots)\\
	\sigma_{> i}X^\bullet\coloneq (\cdots \rightarrow 0\rightarrow X^{i+1}\rightarrow X^{i+2}\rightarrow X^{i+3}\rightarrow \cdots)
\end{gather*}
と定めると
\begin{gather}
	\tau_{\le i}X^\bullet\rightarrow X^\bullet \rightarrow\tau_{>i}X^\bullet\rightarrow \tau_{\le i}X^\bullet[1]\label{canonical}\\
	\sigma_{>i}X^\bullet\rightarrow X^\bullet \rightarrow\sigma_{\le i}X^\bullet\rightarrow \sigma_{> i}X^\bullet[1]\label{stupid}
\end{gather}
が完全三角形となる.


\begin{lemm}
	$X^\bullet\in D^b(\A)$に対し,Zrothendieck群$K(D^b(\A))$の中で
	\[[X^\bullet] = \sum_i(-1)^i[\H^i(X^\bullet)] = \sum_{i}(-1)^i[X^i]\]
\end{lemm}
\begin{proof}
	完全三角形(\ref{stupid})により,
	\[[X^\bullet] = [\sigma_{>i}X^\bullet] + [\sigma_{\le i}X^\bullet]\]
	が成り立つ.$X^\bullet$は有界複体なので,繰り返し適用すれば
	\[[X^\bullet] = \sum_{i\in\ZZ}[X^i[-i]]\]
	完全三角形$X\rightarrow 0\rightarrow X[1]\rightarrow X[1]$を考えれば,$[X] = -X[1]$より$[X^\bullet] = \sum_{i\in\ZZ}(-1)^i[X^i]$\\
同様に完全三角形(\ref{canonical})により,
	\[[X^\bullet] = [\tau_{\le i}X^\bullet] + [\tau_{> i}X^\bullet]\]
繰り返し用いて,
\begin{align*}
	[X^\bullet] &= \sum_{i\in\ZZ}([\Im d_X^i[-i]] + [\Ker d_X^i[-i]])\\
							&= \sum_{i\in\ZZ}((-1)^i(-[\Im d_X^{i-1}] + [\Ker d_X^i])
							\intertext{$0\rightarrow\Im_{X}^{i-1}\rightarrow \Ker d_X^{i} \rightarrow \H^i(X)\rightarrow 0$の短完全列より}
							&= \sum_{i\in\ZZ}(-1)^i[\H^i(X^\bullet)]
\end{align*}
\end{proof}

\begin{lemm}
	$\A$から$\D^b(\A)$への自然な埋め込みによって,
	\[K(\A)\rightarrow K(\D^b(\A))\]
	で群の同型が与えられる.
\end{lemm}
\begin{proof}
	$\A$での短完全列は$\D^b(\A)$での完全三角形を対応させるのでwell-definedである.\\
	\begin{gather*}
		\phi([X])\coloneq [(\cdots\rightarrow 0\rightarrow X\rightarrow 0\rightarrow \cdots)]\\
		\psi([X^\bullet])\coloneq \sum_i(-1)^i[\H^i(X^\bullet)]
	\end{gather*}
	と定めると互いに逆を与えている.
\end{proof}
\end{defn}
