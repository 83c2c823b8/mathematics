\RequirePackage{luatex85}
\documentclass[leqno]{ltjsarticle}
\usepackage{luatexja-fontspec}
\usepackage[top=10truemm,bottom=10truemm,left=20truemm,right=20truemm]{geometry}
\usepackage{luatexja} 
\usepackage{multicol,amsmath,amssymb,mathtools,ascmac,amsthm,amscd,physics,comment,dcolumn,titlesec,mathrsfs,mypkg,tikz-cd,homology}
\usetikzlibrary{arrows.meta}
\titleformat*{\section}{\Large\bfseries}
\setlength{\parindent}{0pt}
\pagestyle{empty}
%\everymath{\displaystyle}
\begin{document}
\section{Affine Varieties}
\mydef{
	$A = k[x_1,\ldots ,x_n]$
	\[Z(T)\coloneq \{P\in \A^n\mid \forall f\in T,\ f(P) = 0\}\]
	$Y\subset  \A^n$ is algebraic set $\defi \exists T\subset A\st Y = Z(T)$
}
\myprop{
	The union of two algebraic sets and intersection of any family of algebraic sets are algebraic set.
}
\begin{proof}
	If $Y_1=Z(T_1), Y_2=Z(T_2)$. then $Y_1\cup Y_2=Z(T_1T_2)$\\
	If $Y_{\lambda}=Z(T_{\lambda})$, then $\bigcap_{\lambda\in\Lambda}Y_\lambda = Z(\bigcup_{\lambda\in\Lambda}T_{\lambda})$
\end{proof}
\mydef{
	Zariski topology on $\A^n \defi \mathcal O = \{Y^c\subset \mathbb A^n \mid Y\colon\text{algebraic set}\}$
	\[I(Y)\coloneq \{f\in A\mid \forall P\in Y,\ f(P)=0\}\]
	$Y(\subset X\colon \text{topological space})$ is irreducible$\defi$ $Y$ cannot be expressed as $Y = Y_1 \cup Y_2 ,\ (\emptyset\subsetneq Y_1,Y_2\subsetneq Y\colon\text{closed})$
}
\mylem{
	Hilbert's basis theorem\\
	$R\colon$ Noetherian $\Longrightarrow R[X]\colon$ Noetherian 
}
\begin{proof}
	Let $I$ be an ideal of $R[X]$\\
	\[J\coloneq\{a_0\in R\mid \exists f\in J\st f(X) = a_0X^d + \cdots + a_d\}\]
	in this definition, $J$ can be confirmed as an ideal of $R$.\\
	\because suppose $a_0 , b_0 \in J$. By definition, there exists $F(X), G(X)\in I \st$
	\begin{gather*}
		F(X) = a_0X^r + \cdots + a_r\\
		G(X) = b_0X^s + \cdots + b_s
	\end{gather*}
	since $I$ is an ideal, $kF(X) \in I$, which means $ka_0\in J$ and $F(X) + X^{r-s}G(X) \in I$, which means $a_0 + b_0 \in J$
	Since $R$ is a Noetherian ring, $J$ is finitely generated. So there exits $a^1,\ldots, a^t \in R\st J=(a^1,\ldots, a^t)$, By definition of $J$, there exists $F_i\ (1\le i \le t)$ whose leading coefficient is $a^i$\\
	For $m\ge 0$, we define $J_m\subset J$ as all leading coefficients of polynomial in I of degree at most $m$. $\ie$
	\[J_m \coloneq \{a_0 \in J\mid r= \deg(f) \leq m,\ f(X) = a_0X^r + \cdots + a_r, \}\]
	$J_m$ can also be verified to be an ideal. Similary $J_m$ is finitedly generated by $a^{m,j}\ (1\le j\le t_m)$ and define similarly $F_{m,j}\ (m<N,\ 1\le j\le t_m)$.\\
	\[I_0\coloneq (F_i\ (1\le i \le t),\ F_{m,j}\ (0\le m< N,\ 1\le j\le t_m))\]
	Obviously $I_0$ is finitely generated and $I_0 \subset I$. If we confirm $I\subset I_0$, this proof is over.\\
	Suppose that there exists a polynomial which doesn't belong to $I_0$. From these polynomial, we take $G$ as the least degree one and let $a$ be a leading coefficient of $G$. Since $a\in J$, there exists $k_1,\ldots,k_t\in R\st a = \sum_{i=1}^{t}k_ia^i$\\
	We consider two cases $\deg(G)\ge N$ and $m = \deg(G)\le N$\\
	\begin{itemize}
		\item In the former case\\
	  We define $H_i(X) \coloneq k_iX^{N-\deg(F_i)}$.
	\[G_0 = G - \sum_{i=1}^{t}H_iF_i\in I\]
	Since $\deg(G_0)\le N$ and $G\in I$, $G_0\in I_0$, this means $G = G_0 + \sum_{i=1}^{t}H_iF_i\in I_0$. Contradiction.\\

	\item In the latter case\\
		Since $a\in J_m$, there exists $k_i\in R\ (1\le i\le t_m)\st a =\sum_{i=1}^{t_m}k_ia^{(m,i)}$\\
		We define $H_i(X) \coloneq k_iX^{N-\deg(F_{m,i})}$.
		\[G_0 = G - \sum_{i=1}^{t}H_iF_{m,i}\in I\]
	Similary we can confirm $G\in I_0$.Contradiction.Therefore $I=I_0$, which means $I$ is finitely generated.
	\end{itemize}
\end{proof}
\mylem{
	Weak Hilbert's Nullstellnsatz\\
	$k\colon$ algebraically closed field, $\mathfrak a\colon$ ideal in $A = k[x_1,\ldots, x_n]$
	\[Z(\mathfrak a)=\emptyset\douchi \mathfrak a= A\]
}
\begin{proof}
	Suppose $\mathfrak a\neq A$. Since $A$ is Noetherian, there exists maximal ideal $\mathfrak m$ that includes $\mathfrak a$.\\
	$A/\mathfrak m$ is isomorphic to some field extension of $k$, but since $k$ is algebraically closed, $A/\mathfrak m = k$. So there exists $a_i\in k\  (1\le i \le n) \st X_i - a_i \in \mathfrak m$.\\
	\[(X_1-a_1,\ldots,X_n-a_n)\subset\mathfrak m\]
However because $(X_1-a_1,\ldots,X_n-a_n)$ is maximal, $\mathfrak m = (X_1-a_1,\ldots,X_n-a_n)$\\
$(a_1,\ldots ,a_n)\in Z(\mathfrak m)\subset Z(\mathfrak a)$\therefore $Z(\mathfrak a)\neq \emptyset$
\end{proof}
\mythm{
	Hilbert's Nullstellnsatz\\
	$k\colon$ algebraically closed field,   $\mathfrak a\colon$ ideal in $A = k[x_1,\ldots, x_n]$\\
	$f \in I(Z(\mathfrak a)) \ie \forall P\in Z(\mathfrak a),\ f(P)=0\Longrightarrow \exists r\in\N,\ f^r \in \mathfrak a$ 
}
\begin{proof}
Since $A$ is a Noetherian ring, there exists $F_i\in A\ (1\le i\le r)\st \mathfrak a = (F_1,\ldots, F_r)$. Suppose that $G\in I(Z(\mathfrak a)) = I(Z((F_1,\ldots, F_r)))$. We define an ideal of $k[X_1,\ldots,X_{n+1}]$ as $\mathfrak b = (F_1,\ldots, F_n,X_{n+1}G-1)$.\\
If $\mathbf a\in Z(\mathfrak a)$, since $G \in I(Z(\mathfrak a))\ie G(\mathbf a)=0$ then $GX_{n+1} - 1 = -1\neq 0$. Therefore $Z(\mathfrak a) = \emptyset$. From previous theorem, $1\in \mathfrak b$. So, there exists $A_1,\ldots, A_r,B\in k[X_1,\ldots,X_{n+1}]\st$
\[\sum_{i=1}^{r}A_iF_i + B(X_{n+1}G - 1) = 1\]
Let $Y = 1/X_{n}$, then there exists $C_1,\ldots, C_r, D\in k[X_1,\ldots, X_n,Y],\ N\in\N\st$
\begin{gather*}
	Y^{-N}\left(\sum_{i=1}^{r}C_iF_i + D(G-Y)\right) = 1\\
	\sum_{i=1}^{r}C_iF_i + D(G-Y) = Y^N\\
	\intertext{Especially $Y = G$}
	G^N = \sum_{i=1}^{r}C_i(X_1,\ldots,X_n,G)F_i \in \mathfrak a
\end{gather*}
\end{proof}
\mylem{
	\[Z(\mathfrak a) = Z(\sqrt{\mathfrak a})\]
}
\begin{proof}
	Since $\mathfrak a \subset \sqrt{\mathfrak a}$, $Z(\mathfrak a) \supset \sqrt{\mathfrak a}$\\
	Let $P\in Z(\mathfrak a)$ and $F \in \sqrt{\mathfrak a}$, then there exists $n\in\N\st F^n(P) = 0$. However since $A$ is integral domain, $F(P)=0$.
\end{proof}
\myprop{
	\begin{enumerate}
		\item[(a)] $T_1\subset T_2 (\subset A) \Longrightarrow Z(T_1) \supset Z(T_2)$
		\item[(b)] $Y_1\subset Y_2 (\subset\A^n) \Longrightarrow I(Y_1) \supset I(Y_2)$
		\item[(c)] $Y_1,Y_2\subset \A^n,\ I(Y_1\cup Y_2) = I(Y_1)\cap I(Y_2)$
		\item[(d)] $\mathfrak a\subset A,\ I(Z(\mathfrak a))=\sqrt{\mathfrak a}$
		\item[(e)] $Y\subset \A^n,\ Z(I(Y)) = \overline Y$
	\end{enumerate}
}
\begin{proof}
	(a),(b),(c) are obvious.\\
	(d)	Let $F\in\sqrt{\mathfrak a}$, then $\forall P\in Z(\mathfrak a),\exists n\in\N\st F^n(P)=0\ie F(P)=0$, which means $F\in I(Z(\mathfrak a))$. Inverse proof is Hilbert Nullstellnsatz.\\
	(e) $Y \subset Z(I(Y))$ is obvious, and $Z(I(Y))$ is closed set, so $\overline Y \subset Z(I(Y))$\\
	Conversely, let $W$ be any closed set containing $Y$, then $W = Z(\mathfrak a)$, some ideal $\mathfrak a$. $Z(\mathfrak a) \supset Y \ie I(Z(\mathfrak a)) \subset I(Y)$. Obviously $\mathfrak a\subset I(Z(\mathfrak a))$, hence $\mathfrak a \subset I(Y)$ ,which means $W=Z(\mathfrak a) \supset Z(I(Y))$. Thus $Z(I(Y))=\overline Y$
\end{proof}

\mycor{
	\[\text{Algebraic sets in }\A^n \overset{1:1}{\longleftrightarrow} \text{Radical ideals in }A\]
	\[\mathfrak a\colon \text{radical ideal}\defi \mathfrak a = \sqrt{\mathfrak a}\]
	\[\text{Algebraic set $Y$ is irreducible}\Longleftrightarrow \text{$I(Y)$ is prime ideal.} \]
}
\begin{proof}
	If $Y$ is irreducible, we show that $I(Y)$ is prime.\\
	If $fg\in I(Y)$, then $Y\subset Z(fg) = Z(f)\cup Z(g)$. $Y = (Y\cap Z(f)) \cup (Y\cap Z(g))$, both being closed subsets of $Y$. Since $Y$ is irreducible, $Y = Y\cap Z(f)$ namely $Y\subset Z(f)$ or $Y\subset Z(g)$, hence $f\in I(Y)$ or $g\in I(Y)$\\
	Conversely, let $\mathfrak p$ be a prime ideal and suppose that $Z(\mathfrak p) = Y_1 \cup Y_2$, then $\mathfrak p = I(Y_1) \cap I(Y_2)$
\end{proof}

\mydef{
	Let $Y\subset \A^n$ be affine algebraic set. We define the affine coordinate $A(Y) \coloneq A/I(Y)$ 
}

\mydef{
	$X\colon$:topological space is Noetherian $\defi$ for any sequence $Y_1\supset Y_2\supset \cdots$ of closed subsets, $\exists r\in \N \st Y_r=Y_{r+1}=\cdots$
}

\myprop{
	$X\colon$ Noetherian topological space, for every nonempty closed subsets can be expressed as following
	\[Y = Y_1\cup Y_2\cdots \cup Y_r\]
	$Y_i\colon$ irreducible. If we require $Y_i\not\supset Y_j$, $Y_i$ are uniquely determined.
}
\begin{proof}
	First we show the existence of representation of $Y$. Let $\mathfrak S$ be the set of nonempty closed subsets of $X$ which cannot be written as a finite union of irreducible closed subset. If $\mathfrak S$ is nonempty, since $X$ is noetherian, there exists minimal element, say $Y$. Then $Y$ is not irreducible. Thus we can write $Y = Y' \cup Y''$, where $Y'$ and $Y''$ are proper closed subsets of $Y$. By minimality of $Y$, each of $Y'$ and $Y''$ can be expressed as finite union of closed irreducible subsets, which is contradiction.\\
	Now suppose $Y=Y_1'\cup \cdots Y_s'$ is another such representation. Then $Y_1'\subset Y_1\cup\cdots Y_r$.so
	\[Y_1' = \bigcup (Y_1'\cap Y_i)\]
	But $Y_1'$ is irreducible, so $Y_1'\subset Y_i$ for some $i$, similarly $Y_1\subset Y_j'$ for some $j$. By condition, we find $Y_1'=Y_1$. Proceeding this processes, we obtain the desired result.
\end{proof}

\mycor{
	Every algebraic set in $\A^n$ can be expressed uniquely as a union of varities, no one containing another.
}
\mydef{
	$X\colon$ topological space.
	We define the dimension of $X$ (denoted $\dim X$) to be the supremum of all integers $n$ such that there exists a chain $Z_0\subsetneq Z_1\subsetneq \cdots \subsetneq Z_n $
}

\mydef{
	In a ring $A$, the height of a pirme ideals $\mathfrak p$ is the supremum of all integers $n$ such that there exists a chain $\mathfrak p_0 \subsetneq \mathfrak p_1\subsetneq \cdots \subsetneq \mathfrak p_n = \mathfrak p $\\
	We define the dimension of $A$ (Krull dimension) of $A$ to be the supremum of the heights of all prime ideals.
}

\myprop{
	If $Y$ is an affine algebraic set, then the dimension of $Y$ is equal to the dimension of its affine coordinate ring $A(Y)$
}
\begin{proof}
	If $Y$ is an affine algebraic set in $\A^n$, then the closed irreducible subsets of $Y$ corresopond to prime ideals of $A=k[X_1,\ldots, X_n]$ containing $I(Y)$, which corresoponds to prime ideals of $A(Y)$.
\end{proof}

\mythm{
	Let $k$ be a field, and let $B$ be an integral domain which is a finitely generated $k$-algebra Then:
	\begin{enumerate}
		\item[(a)]
			The dimension of $B$ is equal to the transcendence degree of the quotient field $K(B)$ of $B$ over $k$
		\item[(b)]
			For any prime ideal $\mathfrak p$ in $B$, we have
			\[\height\mathfrak p + \dim B/\mathfrak p = \dim B\]
	\end{enumerate}
}
\begin{proof}
	
\end{proof}
\end{document}
