\RequirePackage{luatex85}
\documentclass[leqno]{ltjsarticle}
\usepackage{luatexja-fontspec}
\usepackage[top=10truemm,bottom=10truemm,left=20truemm,right=20truemm]{geometry}
\usepackage{luatexja} 
\usepackage{multicol,amsmath,amssymb,mathtools,ascmac,amsthm,amscd,physics,comment,dcolumn,titlesec,mathrsfs,mypkg,tikz-cd,homology}
\usetikzlibrary{arrows.meta}
\titleformat*{\section}{\Large\bfseries}
\setlength{\parindent}{0pt}
\pagestyle{empty}
%\everymath{\displaystyle}
\begin{document}
\section{Affine Varieties}
\mydef{
	$A = k[x_1,\ldots ,x_n]$
	\[Z(T)\coloneq \{P\in \A^n\mid \forall f\in T,\ f(P) = 0\}\]
	$Y\subset  \A^n$ is algebraic set $\defi \exists T\subset A\st Y = Z(T)$
}
\myprop{
	The union of two algebraic sets and intersection of any family of algebraic sets are algebraic set.
}
\begin{proof}
	If $Y_1=Z(T_1), Y_2=Z(T_2)$. then $Y_1\cup Y_2=Z(T_1T_2)$\\
	If $Y_{\lambda}=Z(T_{\lambda})$, then $\bigcap_{\lambda\in\Lambda}Y_\lambda = Z(\bigcup_{\lambda\in\Lambda}T_{\lambda})$
\end{proof}
\mydef{
	Zariski topology on $\A^n \defi \mathcal O = \{Y^c\subset \mathbb A^n \mid Y\colon\text{algebraic set}\}$
	\[I(Y)\coloneq \{f\in A\mid \forall P\in Y,\ f(P)=0\}\]
	$Y(\subset X\colon \text{topological space})$ is irreducible$\defi$ $Y$ cannot be expressed as $Y = Y_1 \cup Y_2 ,\ (\emptyset\subsetneq Y_1,Y_2\subsetneq Y\colon\text{closed})$
}
\mylem{
	Hilbert's basis theorem\\
	$R\colon$ Noetherian $\Longrightarrow R[X]\colon$ Noetherian 
}
\begin{proof}
	Let $I$ be an ideal of $R[X]$\\
	\[J\coloneq\{a_0\in R\mid \exists f\in J\st f(X) = a_0X^d + \cdots + a_d\}\]
	in this definition, $J$ can be confirmed as an ideal of $R$.\\
	\because suppose $a_0 , b_0 \in J$. By definition, there exists $F(X), G(X)\in I \st$
	\begin{gather*}
		F(X) = a_0X^r + \cdots + a_r\\
		G(X) = b_0X^s + \cdots + b_s
	\end{gather*}
	since $I$ is an ideal, $kF(X) \in I$, which means $ka_0\in J$ and $F(X) + X^{r-s}G(X) \in I$, which means $a_0 + b_0 \in J$
	Since $R$ is a Noetherian ring, $J$ is finitely generated. So there exits $a^1,\ldots, a^t \in R\st J=(a^1,\ldots, a^t)$, By definition of $J$, there exists $F_i\ (1\le i \le t)$ whose leading coefficient is $a^i$\\
	For $m\ge 0$, we define $J_m\subset J$ as all leading coefficients of polynomial in I of degree at most $m$. $\ie$
	\[J_m \coloneq \{a_0 \in J\mid r= \deg(f) \leq m,\ f(X) = a_0X^r + \cdots + a_r, \}\]
	$J_m$ can also be verified to be an ideal. Similary $J_m$ is finitedly generated by $a^{m,j}\ (1\le j\le t_m)$ and define similarly $F_{m,j}\ (m<N,\ 1\le j\le t_m)$.\\
	\[I_0\coloneq (F_i\ (1\le i \le t),\ F_{m,j}\ (0\le m< N,\ 1\le j\le t_m))\]
	Obviously $I_0$ is finitely generated and $I_0 \subset I$. If we confirm $I\subset I_0$, this proof is over.\\
	Suppose that there exists a polynomial which doesn't belong to $I_0$. From these polynomial, we take the least degree and 
\end{proof}
\mylem{
	Weak Hilbert's Nullstellnsatz\\
	$k\colon$ algebraically closed field, $\mathfrak a\colon$ ideal in $A = k[x_1,\ldots, x_n]$
	\[Z(\mathfrak a)=\emptyset\douchi \mathfrak a= A\]
}
\begin{proof}
	Suppose $\mathfrak a\neq A$. Since $A$ is Noetherian, there exists maximal ideal $\mathfrak m$ that includes $\mathfrak a$.\\
	$A/\mathfrak m$ is isomorphic to some field extension of $k$, but since $k$ is algebraically closed, $A/\mathfrak m = k$. So there exists $a_i\in k\  (1\le i \le n) \st X_i - a_i \in \mathfrak m$.\\
	\[(X_1-a_1,\ldots,X_n-a_n)\subset\mathfrak m\]
However because $(X_1-a_1,\ldots,X_n-a_n)$ is maximal, $\mathfrak m = (X_1-a_1,\ldots,X_n-a_n)$\\
$(a_1,\ldots ,a_n)\in Z(\mathfrak m)\subset Z(\mathfrak a)$\therefore $Z(\mathfrak a)\neq \emptyset$
\end{proof}
\mythm{
	Hilbert's Nullstellnsatz\\
	$k\colon$ algebraically closed field,   $\mathfrak a\colon$ ideal in $A = k[x_1,\ldots, x_n]$\\
	$f \in I(Z(\mathfrak a)) \ie \forall P\in Z(\mathfrak a),\ f(P)=0\Longrightarrow \exists r\in\N,\ f^r \in \mathfrak a$ 
}
\myprop{
	\begin{enumerate}
		\item[(a)] $T_1\subset T_2 (\subset A) \Longrightarrow Z(T_1) \supset Z(T_2)$
		\item[(b)] $Y_1\subset Y_2 (\subset\A^n) \Longrightarrow I(Y_1) \supset I(Y_2)$
		\item[(c)] $Y_1,Y_2\subset \A^n,\ I(Y_1\cup Y_2) = I(Y_1)\cap I(Y_2)$
		\item[(d)] $\mathfrak a\subset A,\ I(Z(\mathfrak a))=\sqrt{\mathfrak a}$
		\item[(e)] $Y\subset \A^n,\ Z(I(Y)) = \overline Y$
	\end{enumerate}
}
\mydef{
	$X\colon$:topological space is Noetherian $\defi$ for any sequence $Y_1\supset Y_2\supset \cdots$ of closed subsets, $\exists r\in \N \st Y_r=Y_{r+1}=\cdots$
}

\end{document}
